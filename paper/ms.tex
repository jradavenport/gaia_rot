\documentclass[manuscript, letterpaper]{aastex6}
%\documentclass[twocolumn]{aastex6}

\bibliographystyle{aasjournal}

\usepackage{graphicx}
\usepackage[suffix=]{epstopdf}
\usepackage{natbib}
\usepackage{amsmath}
\usepackage{url}
\usepackage{xspace}

% from here: https://github.com/dfm/peerless/blob/master/document/ms.tex#L19-L69
% ----------------------------------- %
% start of AASTeX mods by DWH and DFM %
% ----------------------------------- %
\setlength{\voffset}{0in}
\setlength{\hoffset}{0in}
\setlength{\textwidth}{6in}
\setlength{\textheight}{9in}
\setlength{\headheight}{0ex}
\setlength{\headsep}{\baselinestretch\baselineskip} % this is 2 lines in ``manuscript''
\setlength{\footnotesep}{0in}
\setlength{\topmargin}{-\headsep}
\setlength{\oddsidemargin}{0.25in}
\setlength{\evensidemargin}{0.25in}
\linespread{0.54} % close to 10/13 spacing in ``manuscript''
\setlength{\parindent}{0.54\baselineskip}
%\hypersetup{colorlinks = false}
\makeatletter % you know you are living your life wrong when you need to do this
\long\def\frontmatter@title@above{
\vspace*{-\headsep}\vspace*{\headheight}
\noindent\footnotesize
{\noindent\footnotesize\textsc{\@journalinfo}}\par
{\noindent\scriptsize Preprint typeset using \LaTeX\ style AASTeX6\\
With modifications by David W. Hogg and Daniel Foreman-Mackey
}\par\vspace*{-\baselineskip}\vspace*{0.625in}
}%
\makeatother
% Section spacing:
\makeatletter
\let\origsection\section
\renewcommand\section{\@ifstar{\starsection}{\nostarsection}}
\newcommand\nostarsection[1]{\sectionprelude\origsection{#1}}
\newcommand\starsection[1]{\sectionprelude\origsection*{#1}}
\newcommand\sectionprelude{\vspace{1em}}
\let\origsubsection\subsection
\renewcommand\subsection{\@ifstar{\starsubsection}{\nostarsubsection}}
\newcommand\nostarsubsection[1]{\subsectionprelude\origsubsection{#1}}
\newcommand\starsubsection[1]{\subsectionprelude\origsubsection*{#1}}
\newcommand\subsectionprelude{\vspace{1em}}
\makeatother
\widowpenalty=10000
\clubpenalty=10000
\sloppy\sloppypar
% ------------------ %
% end of AASTeX mods %
% ------------------ %



%    Make Scientific Notation
\providecommand{\e}[1]{\ensuremath{\times 10^{#1}}}

% make the word Kepler italicized
\newcommand{\Kepler}{\textsl{Kepler}\xspace}

\begin{document}

%%%%%%%%%%%%%%%%%%%%%%
\title{Rotating Stars from \Kepler Observed with Gaia DR1}

\shorttitle{Rotating Stars}
\shortauthors{Davenport}

\author{
	James R. A. Davenport\altaffilmark{1,2}
	}

%\altaffiltext{1}{Corresponding author: James.Davenport@wwu.edu}
\altaffiltext{1}{Department of Physics \& Astronomy, Western Washington University, 516 High St., Bellingham, WA 98225, USA}
\altaffiltext{2}{NSF Astronomy and Astrophysics Postdoctoral Fellow}
 

 

%%%%%%%%%%%%%%%%%%%%%%%%%%%%%%
\begin{abstract}
Astrometric data from the recent Gaia Data Release 1 has been matched against the sample of stars from \Kepler with known rotation periods. A total of 1,299 bright rotating stars were recovered from the subset of Gaia sources with good astrometric solutions, most with temperatures hotter than 5000 K. From these sources, 894 were selected as lying near the main sequence using their absolute $G$-band magnitudes. These main sequence stars show a bimodality in their rotation period distribution, centered roughly around a 600 Myr rotation-isochrone. This feature matches the bimodal period distribution from cooler stars with \Kepler, but was previously undetected for solar-type stars due to sample contamination by subgiants.
A tenuous connection between the rotation period and total proper motion is found, suggesting the period bimodality is due to the age distribution of stars within $\sim$300pc of the Sun, rather than a phase of rapid angular momentum loss.
This emphasizes the unique power for stellar populations studies of combining temporal monitoring from \Kepler with astrometric data from Gaia
\end{abstract}



%%%%%%%%%%%%%%%%%%%%%%%%%%%%%%
\section{Introduction}

The\Kepler mission \citep{borucki2010} has enabled the first studies of rotation periods for large ensembles of field stars. The fundamental stellar property of rotation has been measured for over 30k stars using the high cadence \Kepler light curves, tracing the periodic or quasi-periodic modulations in brightness as cool starspots rotate in and out of view \citep{reinhold2013,mcquillan2014}. The seminal work by \citet{skumanich1972} connected stellar rotation and age via angular momentum loss, leading to an age estimating technique known as gyrochronology. At present, ages determined by gyrochronology are accurate to $\sim$10\% in the {\it best} cases (young solar-type stars). With rotation now being routinely measured, it is hoped a new ability of determining robust ages for field stars will be possible by calibrating gyro-isochrones, and  \citep[e.g.][]{angus2015,van-saders2016}. 


Curiously, in studying the rotation periods for M dwarfs in the \Kepler field, \citet{mcquillan2013} discovered a bimodal period distribution, which was subsequently confirmed to exist up to K dwarfs in the \Kepler field by \citet{mcquillan2014}. However, this had never been observed in any other study of stellar rotation periods, including stellar clusters at a variety of ages, nor was it detected in the \Kepler stars at hotter temperatures ($T_{eff} > 5000$). While binary stars and multiple-period systems may be contaminating the rotation period sample for \Kepler field stars, \citet{mcquillan2014} found they could not adequately explain the bimodal period distribution. Currently favored explanations for this feature are 1) a non-continuous age distribution for nearby stars, as was suggested with very nearby Hipparcos stars by \citet{hernandez2000}, or 2) a previously unknown phase of rapid angular momentum loss for low-mass stars, similar to the ``Vaughan-Preston'' gap seen in chromospheric activity indicators \citep{vaughan1980}. As independent age indicators for these field stars are often non-existent, and both scenarios deal with physical mechanisms that are not currently understood with precision, a definitive explanation has not been found.


Astrometric data from the Gaia mission \citep{gaia} can help shed light on this stellar population mystery. By measuring distances via stellar parallax for these rotating stars, the \Kepler--Gaia sample can separate single main sequence dwarfs from binary stars or evolved stars such as subgiants, and will help calibrate fundamental properties of \Kepler stars, such as log(g) \citep{creevey2013}. Galactic kinematics from Gaia will also provide an additional age-proxy, and allow for searches of substructure in field star ages such as from moving groups. The Gaia data will also enable a measurement of the star formation history of the disk from both white dwarf cooling sequences \citep{carrasco2014,gaensicke2015} and color-magnitude diagram models \citep{bertelli1999}.


In this paper I demonstrate the utility of combining temporal properties derived from \Kepler light curves with the preliminary astrometric solutions from  Gaia Data Release 1 \citep[hereafter DR1][]{gaia_dr1}. This combined sample allows improved selection of main sequence stars, and reveals previously undetected structure in the rotation period distribution for solar-type stars.





%%%%%%%%%%%%%%%%%%%%%%
\section{The \Kepler--Gaia Data}
Rotation periods in this study come from \citet{mcquillan2014}, who performed an Auto-Correlation Function analysis of \Kepler stars cooler than 6500 K that had at least $\sim$2 years of observation. The periods recovered from this approach generally agree very well with those found via Lomb-Scargle Periodograms \citep[e.g.][]{reinhold2013,aigrain2015}. Sources with multiple distinct periods, such as from binary systems with two spotted stars \citep[e.g.][]{lurie2015} are detected by \citet{mcquillan2014}, but are not included in the following analysis.

The Gaia Data Release 1 (DR1) provides astrometric positions for over $10^9$ sources from the first year of observation with Gaia. The Tycho-Gaia Astrometric Solution (TGAS) measures improved proper motions and parallaxes for 2 million nearby, bright sources by extending the astrometric solutions from Tycho and Hipparcos. While the TGAS data are not a complete astrometric survey, and have possible systematics in the reported parallaxes \citep{stassun2016}, they represent a significant improvement in the astrometry and kinematics available for stars in the \Kepler field thanks to the precision of Gaia.


Using the CDS X-Match service, I cross-matched the available catalogs from these two surveys. A default cross-match radius of 5 arcsec was used. A total of 33,855 stars were found in cross match between these catalogs, 99.5\% of the sample from \citet{mcquillan2014}. 
A subset of 1,299 objects were recovered in the TGAS sample. Due to the brightness limits of the TGAS sample very few K and M dwarfs were recovered in the TGAS sample. Future releases of Gaia data will ostensibly provide full astrometric solutions for nearly all \Kepler stars. The rotation periods versus stellar effective temperatures for the \Kepler--Gaia matched stars are shown in Figure \ref{fig:all}. 



%tgas = J/ApJS/211/24/table1 \& I/337/tgas
%gaia = J/ApJS/211/24/table1 \& I/337/gaia
\begin{figure}[]
\centering
\includegraphics[width=6in]{fig1.png}
\caption{
Rotation period distribution for 33,855 \Kepler stars from \citet{mcquillan2014} with detections in Gaia DR1 (blue dots). The period bimodality can be seen most clearly for stars with $T_{eff} < 4000$ K as a dearth of sources with periods of $\sim$20 days, but extends to at least $T_{eff}\sim5500$ K according to \citet{mcquillan2014}. The subsample of 1,299 nearby objects found in TGAS are highlighted (red circles), and are mostly hotter stars due to the faint limit of the TGAS sample. 
}
\label{fig:all}
\end{figure}





%%%%%
\begin{figure}[]
\centering
\includegraphics[width=5.3in]{fig2.png}
\caption{Hertzsprung--Russell (HR) diagram using temperatures from \citet{mcquillan2014} and Gaia DR1 $G$-band absolute magnitudes for the 894 stars that pass photometric and parallax quality cuts described in the text. Points are colored by their measured \Kepler rotation periods. Two isochrones from \citet{bressan2012} are shown, with ages of 300 Myr and 4 Gyr (solid and dashed lines).
}
\label{fig:HR}
\end{figure}

%%%%%%%%%%%%%%%%%%%%%%
\section{Selecting Main Sequence Stars}

% (gaia[u'parallax_error'] < 0.4) &
  %            (gaia[u'phot_g_mean_flux_error']/gaia[u'phot_g_mean_flux'] < 0.01))
Though \citet{mcquillan2014} attempted to only measure periods for dwarf starss, the sample of \Kepler--Gaia matched stars contains both main sequence dwarfs and evolved stars (giants and subgiants). Previous studies have shown sometimes significant contamination by giants can affect the implied variability properties for dwarf stars \citep{ciardi2011,mann2012}. Therefore to properly understand the nature of the period distribution, a robust sample of main sequence stars must be selected.

Faint stars were removed by requiring sources have $G$-band flux errors $<1$\%. To ensure accurate distances, and therefore luminosities, parallaxes were required to have errors $<0.4$ mas year$^{-1}$. These cuts left a total of 894 stars from the \Kepler--TGAS matched sample (68\%). The Hertzsprung--Russell (HR) diagram for these stars is shown in Figure \ref{fig:HR}, with each point colored by its \Kepler-measured rotation period. Example isochrones from the \citet{bressan2012} grid are shown for two ages. A systematic offset between the measured absolute $G$-band and the isochrone main sequence of $\sim$0.5 magnitudes is found (Figure \ref{fig:hist}). 

The HR-period diagram in Figure \ref{fig:HR} shows stars with a range of evolutionary states, and could help test post-main sequence angular momentum evolution models \citep[e.g.][]{donascimento2012}. Outliers in this diagram (e.g. the rapidly rotating star at $T_{eff}=4500$ K, $M_G=4.5$ mag, $P_{rot}$=0.652 d, KIC 07957709) are either due to erroneous cross-matching between the \Kepler and Gaia catalogs, or represent interesting systems such as rare binary star configurations or stars that have undergone mergers or ingested giant planets \citep{massarotti2008,tayar2015}. Explaining such outliers is beyond the scope of this work, but are worth further study.



The difference between the 300 Myr isochrone main sequence track and the observed $M_G$ is shown in Figure \ref{fig:hist}. Given the systematic offset of $\sim$0.5 mag, a fairly wide band of stars ($0\le \Delta M_G \le 1$) was selected as being ``close to the main sequence''. This final sample included 440 stars. This simplistic cut is not a robust dwarf--giant, nor single--binary star separator, but serves to select a sample of mostly main sequence stars for the illustrative purpose of this work. More precise selection will require an improved isochrone track, as well as updated parallaxes from the full Gaia DR2.




%%%%%
\begin{figure}[]
\centering
\includegraphics[width=4in]{fig3.png}
\caption{Distribution of differences between the Gaia DR1 $G$ absolute magnitudes and values for main sequence stars from the 300 Myr isochrone shown in Figure \ref{fig:HR}. The range of sources selected near the single star main sequence (red lines) may have additional contamination due to dust extinction and binary companions.}
\label{fig:hist}
\end{figure}




%%%%%%%%%%%%
\section{Extending the Spin-Down Gap}

this feature first observed in rotation period by \citet{mcquillan2013} for M dwarfs

here we show it appears to extended to nearby higher mass stars in the \Kepler field.

the reason this feature did not appear in earlier \Kepler studies is the inability to select main sequence stars from turn-off or subgiants using the KIC


%%%%%
\begin{figure}[]
\centering
\includegraphics[width=5.3in]{fig4alt.png}
\caption{ Rotation period versus temperature for TGAS-matched stars near the isochrone main sequence (blue circles). 
%Distributions of rotation periods for these main sequence selected stars are shown as vertical histograms for 7 bins of temperature (black lines), with each temperature bin having equal numbers of stars. 
As in Figure \ref{fig:all}, the full \Kepler--Gaia matched sample is shown for reference (small black dots). The bimodality in rotation periods previously seen by \citet{mcquillan2014} extends the full range of temperatures in the \Kepler--Gaia main sequence sample shown here.
A \citet{meibom2011} 600 Myr gyrochonology-isochrone roughly traces the bimodality midpoint up to 6000 K (red solid line), but deviates sharply up to $\sim$6200 K (red dotted line). A log-linear extrapolation to of the isochrone from 6000K to 6500 K (red dashed line) continues to track the bimodality to hotter temperatures, and roughly traces a line of constant Rossby number. 
}
\label{fig:gyro}
\end{figure}





\begin{figure}[]
\centering
\includegraphics[width=4in]{fig5.png}
\caption{Residual of log rotation periods about the \citet{meibom2011} 600 Myr isochrone, using the log-linear extrapolation between 6000--6500 K shown in Figure \ref{fig:gyro}. The bimodal rotation period distribution is clear, with peaks at $\Delta \log P_{rot}$ of -0.19 and 0.21 days. A two-Gaussian model was fit to this distribution (blue line).
}
\label{fig:diff}
\end{figure}





%%%%%%%%%%%%%%%%%%%%%%
\section{Discussion}


median distance of K/M stars ($T_{eff} > 4000$ K) is $\sim$216 pc, using isochrone
median distance for the TGAS-matched sample (blue points) is very close, 285 pc, so also sampling just the most local volume.
this points to the effect being localized around us, which explains why it has not been seen in other gyro studies to date (e.g. clusters)
Unfortunately not enough rotation period data across the fully convective boundary ($\sim$3000 K) to tell if this bimodal feature continues.



the bimodality may be another manifestation of the ``Vaughan Preston gap'' \citep{vaughan1980}, e.g. discussed for rotating stars by\citet{kado-fong2016}.
either due to fast evolution through intermediate stellar activity, or could be an age gap. if age, these clumps line up roughly with a 300 Myr and 2 Gyr \citet{meibom2011} gyro-isochrone from cooler than about 6000 K. this similar to the non-continusous star formation history for the local solar neighborhood using Hipparcos astrometry \citep{hernandez2000}. this age explanation was bolstered in \citet{mcquillan2013}, who noted the two populations of rotating M dwarfs had different distributions of proper motions, indicating they belonged to distinct groups of stars. The distribution of total proper motion for stars above and below the modified 600 Myr gyrochrone is shown in Figure \ref{fig:pm}. Stars above the gyrochrone (slower rotators, nominally older) have a median total proper motion of 15.4 mas/yr, while those below (faster rotators, younger)  have a median of 11.3 mas/yr. This difference in kinematics versus rotation period is in the same direction observed by \citet{mcquillan2013}. As the typical error in the total proper motion is $\sim$1.4 mas/yr for this sample, this is a marginal (2.8$\sigma$) difference. Note also that likely contamination of subgiants in the hotter stars may also obscure kinematic differences.

%('median above: ', 15.373856668858519)
%('median below: ', 11.301950163653459)
%('median PM error:', 1.4275539188520987)
%('diff / error = ', 2.8523661708548951)
\begin{figure}[]
\centering
\includegraphics[width=4in]{fig6.png}
\caption{Total proper motion distributions for stars above (rotating slower, older stars) the gyrochrone model shown in Figure \ref{fig:gyro} (black) and below (rotating faster, younger stars) the model (green). Median values of the two distributions are shown (thick lines), which yield a marginal 2.8$\sigma$ difference.
}
\label{fig:pm}
\end{figure}


finally, %gaia + \Kepler useful for determining the age distribution of the nearby field
have shown utility for using Gaia data, combined with detailed light curve statitiscs from \Kepler, to reveal hidden structure in properties of field stars. This combination will be super useful for determining approximate log g values (separating main sequence from sub-giants), and thus more accurate gyrochronology ages \citep{van-saders2013}. also for modeling the star formation history of the entire milky way \citep[e.g.][]{bertelli1999}.



%%%%%%%%%%%%%%%%%
\acknowledgments
Special thanks to Jennifer van Saders, Sean Matt, and Travis Metcalfe for their helpful discussions that motivated publishing this work.

JRAD is supported by an NSF Astronomy and Astrophysics Postdoctoral Fellowship under award AST-1501418.

This research made use of the cross-match service provided by CDS, Strasbourg.

\bibliography{/Users/davenpj3/Dropbox/references}

\end{document}
